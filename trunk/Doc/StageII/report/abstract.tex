\thispagestyle{plain}
\abstract
Single-legged locomotion gait is a hopping motion consisting of alternate flight and stance phases. In such a hopping robot, if the energy lost in friction and impacts is compensated, then control of robot attitude can result in stable hopping motion. In this regard, use of a reaction wheel mechanism as an attitude control actuator is a novel and energy efficient approach.\\

This report discusses the development of the mechanical and electronic systems of a one legged hopping robot. An energy efficient mechanism has been fabricated for experiments to demonstrate a stable running gait as well as in-place hopping using SLOM concept. A test-rig was also built to pivot the robot near its C.G. to perform attitude orientation experiments independently of hopping. Electronics developed during the course of this project consists of an onboard controller, DC motor drivers using MOSFETs and an inertial measurement unit.\\

Dynamics of impulsive systems is significantly different due to its hybrid nature and hence a detailed model of the hopping robot was formulated to study it. This involved simulations for a running gait, in-place hopping as well as exercises in finding the stability basin of the hopper design. Controllabilty of non-linear systems and optimal control techniques were also studied to be implemented later.\\

A full-state attitude feedback system was developed using inertial sensors since the hopping system does not have physical contact with any fixed reference during the flight phase. A kalman filter was designed to attain drift free orientation feedback and was implemented in a real-time embedded controller.

\vspace{1cm}
\begin{description}
  \item[Keywords:] SLOM, offset-mass, reaction wheel, hopping robot, non-linear hopping gait model, inplace hopping, poincare map, attitude control, PID, kalman filter, inertial sensors.
\end{description}