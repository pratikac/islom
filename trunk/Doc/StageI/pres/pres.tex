\documentclass[xcolor=dvipsnames]{beamer}
\usepackage{mathrsfs}
\usepackage{bm}
\usepackage{helvet}
\usepackage{graphicx}
\usepackage[english]{babel}
\usepackage[latin1]{inputenc}
\usepackage{times}

\usetheme{Frankfurt}
\useinnertheme{rounded}
\setbeamercovered{transparent}

\beamertemplatesolidbackgroundcolor{black!5}
\beamertemplatetransparentcovered

\newcommand{\greencol}{\color[RGB]{35,170,30}}
\setbeamerfont{bibfont}{size=\tiny}

\title{Chaos in Cryptography}
%\subtitle{Something catchy here}

\author[Vaibhav and Pratik]{Vaibhav Devanathan\\[0.1in]Pratik Chaudhari}
\institute[IIT Bombay]{IIT Bombay}
\date[\today]{PH 542 - Non Linear Dynamics\\[0.1in]{\footnotesize Prof. P. Paramananda}}

\AtBeginSection[]
{
	\begin{frame}<beamer>
		\frametitle{Outline}
		\tableofcontents[currentsection]
	\end{frame}
}


\begin{document}

\begin{frame}
	\titlepage
\end{frame}

\section[Intro]{Introduction}
\begin{frame}
\frametitle{Cryptography and Chaos}
\begin{block}{Cryptography}
\end{block}

\begin{enumerate}
\item
Message / Signal
\newline
\item
Mask
\newline
\end{enumerate}

\begin{block}{Complexity}
\end{block}
\begin{enumerate}
\item
SDIC
\end{enumerate}
\end{frame}

\begin{frame}
\frametitle{Cryptography}
\begin{block}{Public Key Cryptography}
\end{block}
\vspace{0.2in}
\begin{block}{Example}
\end{block}
\begin{enumerate}
\item
Schematic
\newline
\item
Static Public Key
\newline
\item
Disadvantages
\newline
\item
How can chaos help?
\end{enumerate}
\end{frame}

\section[Types]{Types of Encryptions}
\subsection*{Analog}
\begin{frame}
\frametitle{Implementation}
\begin{block}{Analog System}
\end{block}
\begin{enumerate}
\item
Input is a \alert{Signal}
\newline
\item
\alert{Continuous} Map (3D)
\newline
\item
Transmitted signal is a \alert{Superposition}
\newline
\end{enumerate}
\end{frame}

\subsection*{Digital}
\begin{frame}
\frametitle{Implementation}
\begin{block}{Digital System}
\end{block}
\begin{enumerate}
\item
Input is a \alert{stream of bits}
\newline
\item
\alert{Discrete} Map (1D or 2D)
\newline
\item
Transmitted stream  is a \alert{Sum}
\newline
\end{enumerate}
\end{frame}

\section[Lorenz]{Lorenz equations : Pseudo-random number generators}
\subsection*{Equations}
\begin{frame}
\frametitle{Lorenz System}
\begin{block}{Equations}
\begin{equation*}
\dot{x} = \sigma ( y - x )
\end{equation*}
\begin{equation*}
\dot{y} = x ( \rho - z ) - y
\end{equation*}
\begin{equation*}
\dot{z} = xy - \beta x
\end{equation*}
\end{block}
\begin{block}{Chaotic for,}
$\beta = 2.8$\hfill$\rho = 54.5$\hfill$\sigma = 11.5$
\end{block}
\end{frame}

\begin{frame}
\frametitle{Lorenz Trajectories}
\begin{figure}
\centering
\includegraphics[scale=1.2]{lorenz_plot.pdf}
\end{figure}
\end{frame}

\subsection*{Lena}
\begin{frame}
\frametitle{Lorenz equations on an image}
\begin{enumerate}
\item
Propagate Lorenz equations for $512x512$ steps
\newline
\item
Pseudo-Random sequence with $x$ as \alert{seed}
\newline
\item
\alert{Encryption} -  $B_{(i,j)} \rightarrow B_{(i,j)} + (\; x_{t_{ij}}\;10^3\;mod\;256\;)$
\newline
\item
\alert{Decryption} - $B_{(i,j)} \rightarrow B_{(i,j)} - (\; x_{t_{ij}}\;10^3 \;mod\;256\;)$
\newline
\item
Note the {\em modulo 256} arithmetic
\newline
\item
Can be done for a 3-channel image with $(x, y, z)$ as seed
\end{enumerate}
\end{frame}

\begin{frame}
\frametitle{A Perfect Decryption}
\begin{figure}
\centering
\includegraphics[scale=1]{lena_good.pdf}
\end{figure}
\end{frame}

\begin{frame}
\frametitle{SDIC : Bad Decryption}
\begin{figure}
\centering
\includegraphics[scale=1]{lena_bad.pdf}
\end{figure}
\end{frame}

\section[Sync s-r]{Synchronous Sender-Receiver}
\subsection*{Sync_sr}
\begin{frame}
\frametitle{Synchronous Chaotic System}

\begin{columns}

\column{2in}
\begin{block}{Sender : Lorenz System}
\begin{equation*}
\dot{x_1} = \sigma ( y_1 - x_1 )
\end{equation*}
\begin{equation*}
\dot{y_1} = x_1 ( \rho - z_1 ) - y_1
\end{equation*}
\begin{equation*}
\dot{z_1} = {x_1}y_1 - \beta x_1
\end{equation*}
\end{block}

\column{2in}
\begin{block}{Receiver : Coupled System}
\begin{equation*}
x_2 = x_1
\end{equation*}
\begin{equation*}
\dot{y_2} = x_2 ( \rho - z_2 ) - y_2
\end{equation*}
\begin{equation*}
\dot{z_2} = {x_2}y_2 - \beta x_2
\end{equation*}
\end{block}

\end{columns}

\begin{figure}
\centering
\includegraphics[scale=0.8]{sync_sr.pdf}
\end{figure}
\end{frame}

\subsection*{Analysis}
\begin{frame}
\frametitle{Characteristics}
\begin{block}{Characteristics of Triggered setup}
\end{block}
\begin{enumerate}
\item
\alert{Arbitrary} initial condition
\newline
\item
$x$ is the \alert{trigger} signal
\newline
\item
$[y_1, y_2]$ or $[z_1, z_2]$ are the key-generators
\newline
\item
Sampled at pre-determined intervals
\newline
\item
\alert{Different} initial condition for future communication
\end{enumerate}
\end{frame}

\begin{frame}
\frametitle{Disadvantages}
\begin{enumerate}
\item
Driving function and generators are \alert{dependent}
\newline
\item
Key compromised by interception of driving signal
\newline
\item
Ideally, these should be independent
\newline
\item
A \alert{Slave system} added for enhancing security
\newline
\end{enumerate}
\end{frame}

\section[Sync m-s]{Synchronous Master Slave System}
\subsection*{Sync_ms}
\begin{frame}
\frametitle{Master - Slave system : Schematic}
\begin{figure}
\centering
\includegraphics[scale=1.2]{sync_ms.pdf}
\end{figure}
\end{frame}

\begin{frame}
\frametitle{Master Slave Equations}
\begin{itemize}
\item
Master is again a Lorenz system
\end{itemize}
\begin{columns}
\column{2in}
\begin{block}{Sender's End}
\begin{equation*}
\dot{x_1} = -x_1^3 + y_1
\end{equation*}
\begin{equation*}
\dot{y_1} = -x - x_1 - 8y_1
\end{equation*}
\end{block}
\column{2in}
\begin{block}{Receiver's End}
\begin{equation*}
\dot{x_2} = -x_2^3 + y_2
\end{equation*}
\begin{equation*}
\dot{y_2} = -x - x_2 - 8y_2
\end{equation*}
\end{block}
\end{columns}
\vspace{0.2in}
\begin{itemize}
\item
Both systems are \alert{coupled} and \alert{chaotic}
\end{itemize}
\end{frame}

\begin{frame}
\frametitle{A proof of synchronization}
\begin{itemize}
\item
\begin{equation*}
\dot{x}^{\ast} = \dot{x_1} - \dot{x_2} = - (x_1^3 - x_2^3 ) + y1 - y2
\newline
\end{equation*}
\begin{equation*}
\dot{x}^{\ast} = -x^{\ast} (x_1^2 + x_1 x_2 + x^2) + y^{\ast}
\end{equation*}
\item
\begin{equation*}
\dot{y}^{\ast} = \dot{y_1} - \dot{y_2} = - x^{\ast} - 8 y^{\ast}
\end{equation*}
\vspace{0.05in}
\item
\alert{Lyapunov function} is taken as $V = \frac{1}{2}({x^{\ast}}^2 + {y^{\ast}}^2 )$
\end{itemize}

\begin{figure}
\centering
\includegraphics[scale=0.7]{y1_y2.pdf}
\end{figure}
\end{frame}

\subsection*{Diff}
\begin{frame}
\frametitle{Key Differences}
\begin{block}{Synchronous system}
\end{block}
\begin{enumerate}
\item
Sender \alert{parameters} are the key for cipher
\vspace{0.05in}
\item
Sender and receiver have \alert{same form}
\vspace{0.05in}
\end{enumerate}
\begin{block}{Master - Slave system}
\end{block}
\begin{enumerate}
\item
\alert{Slave dynamics} is the key for cipher
\vspace{0.05in}
\item
Different Slave dynamics for sender and receiver
\vspace{0.05in}
\item
Extremely difficult to break even if Master is compromised
\end{enumerate}
\end{frame}

\section[Code Breaking]{Are Chaotic Ciphers non-breakable?}
\subsection*{failures}
\begin{frame}
\frametitle{Fallibilities}
\begin{block}{Classical discrete systems}
\end{block}
\begin{enumerate}
\item
Limited \alert{key-space}
\newline
\item
Redundancy in plain text - \alert{probable words}
\newline
\begin{itemize}
\item
Reverse engineering cipher text
\end{itemize}
\vspace{0.1in}
\item
Brute force attack, \alert{No need of a key!}
\end{enumerate}

\begin{block}{Continuous systems}
\end{block}
\begin{enumerate}
\item
\alert{Trade-off} between SNR and security
\newline
\item
Round off errors
\newline
\end{enumerate}
%\alert{Befriending the sender?}
\end{frame}

\subsection*{matthew}
%Think about deleting this frame
\begin{frame}
\frametitle{Matthew's Chaotic Function}
\begin{equation*}
x_{n+1} = (\beta + 1)\;(1 + \frac{1}{\beta})^{\beta}\;x_n\;(1 - x_n)^{\beta}
\end{equation*}
where $\beta = 2.53$ and $x_n = 0.45$.
\newline
\begin{enumerate}
\item
Length of repeating cycles for 8 significant digits
\newline
\item
Repeated data in plain text makes it worse
\begin{figure}
\centering
\includegraphics[scale=0.8]{matthew.pdf}
\end{figure}
\end{enumerate}
\end{frame}

\subsection*{Lorenz better}
\begin{frame}
\frametitle{Making the Lorenz Cipher even better...}
\begin{enumerate}
\item
\alert{$\Delta t \geq 0.0245$} - The Great Escape
\newline
\item
Correlation between $\Delta x$, $\Delta y$ and $\Delta z$ for small $x$, $y$ and $z$
\newline
\item
Conversion to integers for \alert{modulo arithmetic}
\begin{equation*}
x_{n+1} = (((round (x + \Delta x)\;mod\;m )+m)\; mod\;m) + 1
\end{equation*}
\begin{equation*}
\Delta x = \sigma\;(x_n - y_n)\;\Delta t
\end{equation*}
\vspace{0.01in}
\item
\alert{Diffusion} of plain text in cipher text
\end{enumerate}
\end{frame}

\begin{frame}
\frametitle{An Improved Encryption on Lena}
\begin{block}{Change of initial condition}
\begin{enumerate}
\item
$(0.2030, 0.56680, 0.76990)$ $\rightarrow$ $\delta x = 10^{-7}$
\newline
\item
$(0.2030, 56.680, 76.990)$ $\rightarrow$ $\delta x = 10^{-8}$
\end{enumerate}
\end{block}
\vspace{0.2in}
\begin{block}{Change of time sampling}
\begin{enumerate}
\item
$\Delta t = 0.001 s$ $\rightarrow$ $\delta x = 10^{-6}$
\newline
\item
$\Delta t = 1 s$ $\rightarrow$ $\delta x = 10^{-7}$
\end{enumerate}
\end{block}
\end{frame}	

\begin{frame}
\frametitle{Further Reading}
\usebeamerfont{bibfont}
\begin{enumerate}
\item
R. He and P. G. Vaidya, {Implementation of chaotic cryptography with chaotic synchronization}, {\em American Physical
Society}, Review E, Vol. 57, 2, Feb. 1998.
\newline
\item
P. Lee, S. Pei and Y. Chen, {Generating Chaotic Stream Ciphers Using Chaotic Systems}, {\em Chinese Journal of Physics},
Vol. 41, 6, Dec. 2003.
\newline
\item
Q. Lawande, B. Ivan and S. Dhodapkar, {Chaos Based Crytography: A New Approach to Secure Communications},
{\em BARC Newsletter}, No. 258, July 2005.
\newline
\item
H. Kangwei and T. Lih, {Chaos and Cryptography : Applications and Analysis}, {\em Course Project USC 30001}
\newline
\item
D. Wheeler, {Problems with Chaotic Cryptosystems}, {\em Article published in Cryptologia}, July 1989.
\newline
\item
J. Carroll, J. Verhagen, P. Wong, {Chaos in Cryptography : The Escape from the Strange Attractor},
{\em Article published in Cryptologia}, Jan. 1992.
\end{enumerate}
\end{frame}	


\end{document}


