\newpage
\addcontentsline{toc}{chapter}{Abstract}
\chapter*{Abstract}
\label{chap:abs}
{
\normalsize
Single-legged locomotion gait is a hopping motion consisting of alternate flight and stance
phases. In such a hopping robot, if the energy lost in friction and impacts is compensated,
then along with control of robot attitude we can have stable hopping motion.\\

An offset-mass hopping robot
is a novel idea in the realm of single-legged
robots. These robots have an inherent tendency to leap forward due to the offset mass which eases the 
requirement of effort from the actuators. A reaction wheel is necessary to ensure that we can have a stable
gait for different horizontal velocities and initial conditions. This project aims to build a prototype of a
SLOM one legged hopper and its reaction wheel mechanism to demonstrate a stable 2D hopping gait.\\

The structure of this report is as follows,
\begin{itemize}
\item
Chapter \ref{chap:intro} introduces the one legged hopper problem along with a discussion of previous work in this area. It focusses on the different energy pumping mechanisms and balancing techniques used.
\item
Chapter \ref{chap:problem} details previous work done at IIT Bombay on the SLOM hopper and motivates the
problem statement for the current project through this discussion.
\item
Chapter \ref{chap:mech_design} talks about two different mechanical designs for the SLOM hopper. A qualitative
analysis of both the designs is provided along with pointers for the choice of mechanisms of the final design from among them.
\item
We look at three major aspects of hopper design in Chapter \ref{chap:sizing}. It provides an analysis of the
2 mass problem, frequency modes and reaction wheel stabilization to ultimately choose the design values for
the masses and the motors.
\item
A brief introduction of the embedded platform on the hopper is provided in Chaper \ref{chap:embedded}. It
provides results of the testing of a Kalman filter for the IMU using actual sensors. We also look at
velocity control of a motor with quadrature encoders at the end.
\end{itemize}

\textit{\textbf{Keywords:\\} SLOM, offset-mass, energy-pumping, hopping robot, height control, reaction wheel,
attitude control, Kalman filter, inertial measurement unit}
