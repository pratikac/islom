\chapter{Embedded System}
\label{chap:embedded}
This chapter details the development of the embedded system necessary for controlling and actuating the hopper. It consists of two
major parts yet,
\begin{enumerate}
\item
Micro-controller and RF interface
\item
Inertial Measurement Unit (IMU)
\end{enumerate}

\section{Micro-controller}
This microcontroller chosen for this system is a Microchip dsPIC33F64MC804. It can run at 40 MIPS with an onboard flash memory of
64 KB along with a 16 KB SRAM. There are a host of integrated peripherals like Serial Peripheral Interface (SPI), UART, Analog to
Digital convertor (ADC), Digital to Analog Convertor (DAC) and Timers to generate Pulse Width Modulation (PWM) that can be used.
Other major features provided with it are the Direct Memory Access controller (DMA) which can transfer memory from one peripheral
to other without CPU intervention and the high multiplexing of its IO pins. The latter enables almost all IO pins to be assigned
to any of the above mentioned peripherals (except analog pins) and greatly simplifies design. Pickit 2 is a programmer cum
debugger that had been ordered for programming the micro-controller.\\

XBee modules using the Zigbee protocol are used to form a wireless link with the embedded system on the hopper. The range of these
devices is quite sufficient for indoor uses (about 100 m). The interface on the microcontroller side is through UART and a custom
made FT232 module will be used to interface it with the base station to gather telemetry. A python module to grab and plot this 
telemetry from the virtual serial port of FT232 is in development.\\

The current development board contains one motor driver and its adjoining encoder port. The final module will contain two motor
drivers and encoder arrangements for the pinion and the reaction wheel motors.

\section{Inertial Measurement Unit}

